\chapter{Algemeen besluit}
\vspace{-3cm}

%\iffalse
%Zowel tijdens het uitwerken van de stageopdracht als tijdens het schrijven van deze bachelorproef werd heel wat bijgeleerd. Met de nieuw verworven kennis werd het implementeren van een Profile Manager binnen de CX Social webapplicatie mogelijk. De kern van deze feature was het dynamisch annoteren van afbeeldingen. 
Tijdens het uitwerken van deze bachelorproef werd kennis gemaakt met heel wat nieuwe technologie\"{e}n. Met de nieuw verworven kennis werd het implementeren van een Profile Manager binnen de CX Social applicatie mogelijk. %Naast het werken met bibliotheken zoals Fabric.js werd  technologie\"{e}n als React en Node.js 

Aangezien de Profile Manager een redelijk uitgebreide feature is, werden heel wat doelstellingen gesteld. Naast het uitwerken van een data model en het schrijven van de nodige testen, waren het annoteren van de afbeeldingen en deze periodiek uploaden naar sociale media het belangrijkst. Het implementeren van dit alles werd gespreid over de volledige periode dat stage gelopen werd. Hoewel deze feature zeer uitgebreid is en heel wat functionaliteit bevat, werden de meeste doelstellingen behaald. Zeker de belangrijkste elementen zijn volledige uitgewerkt. Onderzoek naar een bibliotheek die het manipuleren van afbeeldingen vereenvoudigen was noodzakelijk aangezien dit de kern was van de stageopdracht. Na onderzoek bleek Fabric.js de perfecte bibliotheek te zijn en kon de nodige functionaliteit ge\"{i}mplementeerd worden.
% Dit houdt in dat gebruikers een thema kunnen aanmaken en dit later in te stellen voor \'{e}\'{e}n of meerdere profielen. Een thema kan een zowel een omslagfoto en profielfoto bevatten en aan de omslagfoto kunnen annotaties toegevoegd worden. Annotaties kunnen zowel tekst als dynamische KPI's zijn waardoor gebruikers hun omslagfoto zeer uitgebreid kunnen personaliseren.

De Profile Manager kan voor veel bedrijven en ondernemingen een meerwaarde zijn voor hun sociale media management. Vooral het feit dat de KPI's up-to-date gehouden worden, maakt het mogelijk om hun klanten vitale informatie te bezorgen. Doordat thema's ingesteld kunnen worden om actief te zijn op verschillende tijdstippen wordt heel wat werk bespaard. 

Niet enkel op technisch vlak maar ook op sociaal vlak werd heel wat bijgeleerd. Door het functioneren in een team, werden heel wat nieuwe inzichten verworven. Rekening houdend met visies en expertise van teamgenoten, wordt het mogelijk om problemen op een totaal andere manier aan te pakken. Naast een \textit{second opinion} over reeds geschreven code, kon ook om hulp gevraagd worden tijdens het aanpakken van venijnige problemen. Wanneer een hardnekking probleem niet opgelost kon worden na zelfstandig onderzoek kon altijd hulp ingeroepen worden. Ook door de mogelijkheid tot persoonlijke inbreng, werd het werken in een team zeer aangenaam. 

Wekelijkse meetings droegen bij tot het effici\"{e}nt en correct uitwerken van de annotatie feature binnen de Profile Manager. In samenwerking met de designer werden zowel veranderingen in het ontwerp als in de functionaliteit uitgevoerd om kleine beperkingen van de bibliotheek te omzeilen. Ook dit zorgde voor een andere kijk op de implementatie aangezien niet enkel naar de code maar ook naar het ontwerp en uiteindelijk de gebruikservaring werd gekeken. 

Na 12 weken intensief werken aan de Profile Manager, kon de beta versie online geplaatst worden. Deze versie zal uitgebreid getest worden door het support team en wanneer geen problemen meer gevonden worden, kan de feature online geplaatst worden. Eens beschikbaar, zal het zeker een meerwaarde vormen voor verschillende klanten. Het feit dat deze feature actief gebruikt zal worden door bedrijven is een directe beloning voor het verrichte werk.  

















%// Zeer handige feature
%// Mogelijkheid tot persoonlijke inbreng. Vooral bij de kleinere functionaliteiten en hoe zaken geimplementeerd moeten worden
%// De meeste doelstellingen zijn bereikt van het opgestelde actieplan. Enkel de integratie in de crisisplans (en welcome messages) zijn er niet 
%\fi