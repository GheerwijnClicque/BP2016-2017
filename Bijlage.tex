
\appendix
\chapter{Bijlage}
\iffalse
\section{Naive Bayes op basis van ingredi\"{e}nten }
\begin{lstlisting}
# inladen van de data
ReceptenJO.binair <- read.table("ReceptenJO.binair.csv", 
	header=TRUE, quote="\"")

# SPLITSEN VAN DE DATASET IN TEST EN TRAIN DATA
nfolds <- 4

id <- rep(1: nfolds,length.out=dim(ReceptenJO.binair)[1])
id <- sample(id,dim(ReceptenJO.binair)[1])
id.test <- (1:dim(ReceptenJO.binair)[1])[id==1]
id.train <- (1:dim(ReceptenJO.binair)[1])[id!=1]

Recepten.test <- ReceptenJO.binair[id.test,]
Recepten.train <- ReceptenJO.binair[id.train,] 

Bernoulli_NB <- function(ReceptenJO.train, ingredienten){
# SCHRIJVEN VAN NAIVE BAYES MODEL OBV INGREDIENTEN
# De input is een binaire vector 
# die de aan- & afwezigheid van ingredienten weergeeft
# De output is het keukentype waarin het recept geclassificeerd werd
  
# kans dat een ingredient in een gegeven keuken voorkomt
# pseudocounts = 0.01
Recepten.train[1:372] <- as.matrix(Recepten.train[1:372]) + 0.01 
IngredientenJO <- Recepten.train[,1:373] 
likelihoodIngred <-aggregate(.~cuisine, IngredientenJO, mean)
likelihoodIngred$cuisine <- NULL

# enkel de ingredienten nodig als input, niet het keukentype
# en de nutritionele waarden: 
# kolom 1 tem 372 wordt dus geselecteerd
input <- as.matrix(ingredienten[1:372]) 

# initialisatie: er zijn 10 keukentypes met 372 ingredienten    
likelihood <- matrix(0,10,372) 
  for (j in 1:10){
    for (k in 1:372){
      likelihood[j,k] = input[k]* likelihoodIngred[j,k] + 
												(1-input[k])*(1-likelihoodIngred[j,k])
    }
  }
	
# matrix met dimensies 10x372
loglikelihood <- log(likelihood)
# wanneer likelihood = 1, is de loglikelihood = 0
# DUS corrigeren met pseudocounts 0.01 => log(0.99) ipv log(1)
loglikelihood[is.na(loglikelihood)] <- log(0.99) 
# loglikelihood geeft nu per keukentype de kans dat de ingegeven
# ingredientenCOMBINATIE (input) in die bepaalde keuken voorkomt 
# DUS matrix met dimensies 10x1
  
loglikelihood <- apply(loglikelihood,1,sum) 

keukentypes <- levels(Recepten.train$cuisine)
maximum <- max(loglikelihood)
index <- loglikelihood==maximum
keukentype <- keukentypes[index]
  
  return(keukentype)
}


# TESTEN OP TESTDATA + CONFUSIONMATRIX OPSTELLEN
voorspelling <- matrix(0,length(Recepten.test[,1]), 1)

for (i in 1 : length(Recepten.test[,1])){
  voorspelling[i] <-Bernoulli_NB(Recepten.train,Recepten.test[i,])
}

confusionTable <- table(voorspelling, as.matrix(Recepten.test$cuisine))

\end{lstlisting}
\fi