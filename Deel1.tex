\chapter{Voorstelling van het stagebedrijf}
\vspace{-3cm}
\section{Ontstaan en geschiedenis}

Engagor werd opgericht in 2011 door Folke Lemaitre (ex-werknemer van Netlog) met de bedoeling interactie met klanten, via sociale media, gemakkelijker te maken. Tegenwoordig is sociale media niet meer weg te denken uit het dagelijkse leven, niet bij particulieren noch bij bedrijven. Dit resulteert in een zeer groot klantenbestand met klanten zoals Telenet, NMBS, AirBnb, \ldots 

Sinds 2015 maakt Engagor deel uit van Clarabridge. Dit Amerikaanse software bedrijf ontwikkelt software om automatisch online data te verzamelen, te categoriseren en te rapporteren. Gebruikelijk is dit data van social media zoals Facebook en Twitter. Maar andere platformen zoals YouTube, Instagram, Vimeo en Tumbler kunnen ook beheerd worden. Kort samengevat zijn ze dus gespecialiseerd in software voor Customer Experience Management (CEM). Het gaat hier dus vooral om feedback over interacties die plaatsvinden tussen de klanten en het bedrijf. Zo kunnen bedrijven hun strategie aanpassen om hun klanten zo goed mogelijk van dienst te zijn \cite{bp1}.  

Na de overname werd Engagor onderdeel van CX Social (CX staat hier voor Consumer Experience), een van de vier producten van Clarabridge. De andere drie zijn CX Survey, CX Studio en CX Designer. Met CX Survey kunnen surveys aangemaakt en verwerkt worden. De verwerking van surveys gebeurt onder andere met behulp van tekst en sentiment analyse. CX Studio is een \textit{dashboarding tool} waarmee gebruikers hun CEM proces kunnen stroomlijnen. Als laatste is er ook nog CX Designer dat voor data modelering gebruikt wordt \cite{Clarabridge.com}. 

Clarabridge heeft momenteel ongeveer 250 werknemers, verspreid over de hele wereld. Het hoofdkantoor bevindt zich in Reston, Virginia, Amerika. Ook zijn er kantoren in Gent, Larkspur (California), Londen, Barcelona en Singapore \cite{ClarabridgeOffices}. Ook hebben ze support teams op deze locaties alsook in India. Zo kunnen ze 24/5, voor sommige klanten zelfs 24/7, hun klanten bijstaan.

Voor mijn stage werk ik aan CX Social. Dit is dus de \textit{customer care tool} voor sociale media. Gebruikers kunnen via deze web applicatie snel en gemakkelijk antwoorden op vragen en/of klachten op hun sociale profielen zoals Facebook, Twitter en Instagram. Dit is voor grote bedrijven natuurlijk zeer handig aangezien zij dagelijks gigantisch veel berichten op hun profielen moeten bewerken. \'{E}\'{e}n gemeenschappelijke plaats waaruit alle profielen kunnen onderhouden worden, is dus onontbeerlijk. Het team bestaat uit ongeveer 35 werknemers. Zij werken dus constant aan het verbeteren en onderhouden van deze tool.  Binnen het team werken de meesten als Full stack software engineer. Zij werken dus zowel aan de backend van de web applicatie als aan de API en de frontend. Het team wordt vervolledigd door mensen die zich bezighouden met \textit{data science} en design. \textit{Data scientists} werken onder andere met \textit{machine learning} en statistieken. Het belang hiervan valt in een \textit{social media monitoring tool} niet te onderschatten. Ook het design is zeer belangrijk tijdens het ontwikkelen van een web applicatie. Telkens een nieuw \textit{feature} toegevoegd moet worden, moet hiervan een ontwerp gemaakt worden om dit nieuwe stukje zo goed mogelijk in het geheel te implementeren. De designers moeten er natuurlijk ook voor zorgen dat hun ontwerpen overeenkomen met alle producten die Clarabridge aanbiedt. Ook moet alles wat ze ontwerpen zo gebruiksvriendelijk mogelijk gemaakt worden \cite{EngagorTeam}. 

Wanneer nieuwe \textit{features} ge\"{i}mplementeerd moeten worden, wordt dit ofwel door \'{e}\'{e}n iemand uitgewerkt of door enkele personen uit het team. Tijdens het ontwikkelen kan feedback en hulp van collega`s ingeroepen worden waardoor iedereen wel altijd betrokken is bij verschillende projecten. Dit zorgt er voor dat niet enkel degene die het project toegewezen krijgt, weet hoe alles in elkaar zit.  Zo is iedereen op elk moment op de hoogte van veranderingen aan de code base. 

Op Figuur~\ref{fig:Organigram} is de organisatie binnen het CX Social team te zien. Dit organigram is niet volledig accuraat want de positie van Solutions Engineer moet hier nog aan toegevoegd worden maar het schept een zeer goed beeld van de verschillende functies binnen CX Social. 

\begin{figure}[H]
	\centering
	\includegraphics[width=1\textwidth]{Figuren/Organigram.png}
	\caption{Organigram CX Social .} %\cite{Pouladzadeh}
	\label{fig:Organigram}
\end{figure} 


\section{CX Social}
Elk product, aangeboden door Clarabridge, is toegespitst op de analyse van customer feedback. CX Social is de social media monitoring, management en analyse tool die gebruikt wordt door verschillende grote merken om hun klanten beter te ondersteunen. Op \'{e}\'{e}n gemeenschappelijke plaats kunnen verschillende profielen beheerd worden. Zo kunnen onder meer berichten van klanten beantwoord worden, statistieken bekeken worden of kan de content van deze pagina's aangepast worden. 

Engagor maakt in hun webapplicatie gebruik van PHP, JavaScript en CSS. Het grootste deel is geschreven in PHP met het DooPHP \textit{framework}. Dit \textit{framework} is momenteel niet meer in ontwikkeling maar werd aangepast voor gebruik binnen de Engagor webapplicatie waardoor verdere ontwikkeling van DooPHP door de ontwikkelaars niet onmiddellijk nodig is \cite{DooPHP}. 

In de frontend wordt vooral gebruik gemaakt van React (ook gekend als React.js/ReactJS). Dit is een open-source JavaScript bibliotheek, ontwikkeld door Facebook, waarmee zeer gemakkelijk gebruikers interfaces aangemaakt kunnen worden. Met React kunnen componenten gecre\"{e}erd worden, waarin wordt vastgelegd wat gerenderd moet worden voor de gebruiker. React zal ervoor zorgen dat de componenten effici\"{e}nt en correct ge\"{u}pdatet worden wanneer data verandert. Aan deze componenten kunnen ook parameters meegegeven worden en op basis hiervan kunnen views aan de gebruiker getoond worden \cite{React}. 

Naast PHP en JavaScript frameworks wordt ook gebruik gemaakt van MySQL, Redis, memcached, ElasticSearch en RabbitMQ. 

CX Social is dus, zoals eerder vermeld, een social media monitoring, management en analyse tool. Zo kunnen klanten op de webapplicatie een centrale inbox raadplegen (zie Figuur~\ref{fig:Inbox}). Hier worden berichten per profiel verzameld en kunnen medewerkers reageren op de vragen/opmerkingen van klanten. 

\begin{figure}[H]
	\centering
	\includegraphics[width=1\textwidth]{Figuren/Inbox.png}
	\caption{Inbox CX Social \cite{EngagorScreenshots}} %\cite{Pouladzadeh}
	\label{fig:Inbox}
\end{figure} 

Via de `Insights' kunnen bedrijven ook de statistieken van hun pagina's bijhouden. Zo zijn de totale mentions van een \textit{topic} te zien alsook de \textit{mentions} per pagina. Dit alles wordt zeer intu\"{i}tief  weergegeven in de vorm van cirkeldiagrammen en tijdlijnen. Hier wordt dan ook het sentiment berekend. Dit geeft een aanduiding van het percentage van de mentions die een positieve, negatieve of neutrale connotatie hebben. Ook trends (veelgebruikte woorden, hashtags  etc.) en foto's die binnen een topic voorkomen zijn hier te vinden (zie Figuur~\ref{fig:Insights}).  

\begin{figure}[H]
	\centering
	\includegraphics[width=1\textwidth]{Figuren/Insights.png}
	\caption{Insights CX Social \cite{EngagorScreenshots}} %\cite{Pouladzadeh}
	\label{fig:Insights}
\end{figure} 

Andere statistieken zijn ook te vinden op de `Performance' pagina. Hier is bijvoorbeeld te zien hoe lang het duurde voor actie werd ondernomen, hoeveel unieke gebruikers geholpen werden of hoeveel antwoorden gemiddeld gegeven werden op een mention (zie Figuur~\ref{fig:Performance}).

\begin{figure}[H]
	\centering
	\includegraphics[width=1\textwidth]{Figuren/Performance.png}
	\caption{Performance CX Social \cite{EngagorScreenshots}} %\cite{Pouladzadeh}
	\label{fig:Performance}
\end{figure} 


Niet enkel berichten kunnen beheerd worden via de webapplicatie. Er kan ook gepost worden op profielen. Dit kan per topic en per profiel gebeuren. Posts kunnen ook gepland worden of opgeslagen worden als \textit{canned response}. Dit zijn standaard antwoorden die gebruikt kunnen worden tijdens het communiceren met klanten \cite{EngagorApp}.
 
\chapter{Stageopdracht}
\vspace{-3cm}
Als stageopdracht dient een nieuwe feature toegevoegd te worden aan de webapplicatie: een \textit{profile manager}. Deze \textit{profile manager} moet gebruikers in staat stellen de profielfoto en/of omslagfoto van een profiel te veranderen. Dit gebeurt aan de hand van thema's. Een thema kan aangemaakt worden voor een specifieke service, bijvoorbeeld Facebook of Twitter. Voor Facebook kan een thema een omslagfoto en een profielfoto bevatten terwijl een thema voor Twitter naast de foto's ook de link kleur, naam en bio kan bevatten (zie Figuur~\ref{fig:EditTheme}). Het moet ook mogelijk zijn voor de klanten om deze afbeeldingen te passen. Zo kunnen ze tekst toevoegen aan de afbeelding of hun eigen \textit{Key Performance Indicators} (KPI) er op plaatsen. Aangezien de omslagfoto's van Facebook en Twitter niet dezelfde dimensies bezitten en de foto ook niet op volledige grootte aan de gebruikers kan getoond worden, zullen ook de afbeeldingen herschaald moeten worden.

Verschillende thema's moeten ook aan een planning toegevoegd kunnen worden. Een klant kan instellen dat een bepaald thema ofwel binnen de \textit{business hours} gebruikt wordt ofwel tijdens een bepaalde periode ofwel tijdens beide. Een speciaal thema voor bijvoorbeeld Kerstmis kan dan ingesteld worden om tijdens de kerstperiode ingesteld te zijn op het profiel. 

De webapplicatie bezit ook \textit{Crisis Plans}. Dit zijn instellingen die bij een probleem actief gesteld kunnen worden. Eens actief wordt een waarschuwing getoond, rechten van gebruikers verandert of een to-do lijstje opgesteld met stappen om het probleem op te lossen. Via een \textit{Crisis Plan} moet ook een thema ingesteld kunnen worden. Een gebruiksscenario hiervoor zou bijvoorbeeld kunnen zijn: het instellen van een specifieke omslagfoto en avatar wanneer een bedrijf kampt met een technische storing. 

\begin{figure}[H]
	\centering
	\includegraphics[width=0.6\textwidth]{Figuren/EditThemeMockup.png}
	\caption{Mockup van de editeerpagina \cite{EditThemeMockup}} %\cite{Pouladzadeh}
	\label{fig:EditTheme}
\end{figure} 

\chapter{Omschrijving bachelorproef}
\vspace{-3cm}
Deel van de stageopdracht is het verwerken van afbeeldingen om als omslagfoto of avatar te gebruiken. Hierbij is het de bedoeling dat klanten afbeeldingen kunnen uploaden die dan als omslagfoto en/of avatar zullen dienen.  Een thema, bestaande uit deze afbeeldingen en in sommige gevallen nog enkele andere variabelen zoals link kleur, info of naam van het profiel, kunnen ingesteld worden om op bepaalde tijdstippen actief te zijn. Klanten moeten in staat zijn tekst en KPI's op deze afbeeldingen te plaatsen vanuit de webapplicatie. De afbeeldingen moeten samengesteld kunnen worden door de gebruiker maar moeten nadien ook opnieuw gerenderd kunnen worden met aangepaste KPI's. Dit moet dus op de server gebeuren. 

Om afbeeldingen te annoteren is redelijk wat functionaliteit nodig. Ingegeven tekst moet eender waar op een afbeelding geplaatst kunnen worden maar moet ook gestyled kunnen worden. Gebruikers moeten alle basiseigenschappen van een stuk tekst kunnen aanpassen. Lettertype, kleur, lettergrootte, gewicht en uitlijning moeten allemaal ingesteld kunnen worden door gebruikers. Dit moet zowel clientside als serverside mogelijk zijn. Bibliotheken en technologie\"{e}n zullen onderzocht worden om dit te kunnen verwezenlijken.


  


