\chapter{Samenvatting}
\vspace{-3cm}
CX Social is de customer care tool voor sociale media waar ik van februari 2017 tot mei 2017 een profile manager uitwerkte. Enkele features die CX Social aan hun klanten aanbiedt zijn: een gecentraliseerde inbox,  een dashboard en een overzichtspagina van de statistieken per profiel. Zo kunnen bedrijven hun klanten zo snel en efficiënt mogelijk bijstaan. Net zoals alle andere features in de web applicatie van CX Social, moet de profile manager klanten helpen bij het beheren van hun sociale media profielen. Het doel is om een gemeenschappelijke plaats te hebben waar klanten de profiel-en omslagfoto van hun profielen kunnen beheren.

Naast het selecteren van afbeeldingen moeten gebruikers in staat zijn om annotaties toe te voegen aan de omslagfoto. Dit houdt in dat zowel tekst als specifieke prestatie-indicatoren op een afbeelding geplaatst kunnen worden. Met behulp van deze indicatoren kan gemakkelijk informatie aan klanten meegedeeld worden. Mogelijke indicatoren zijn cijfers die aanduiden hoe lang het gemiddeld duurt voor vragen beantwoord worden of hoeveel gebruikers het bedrijf al geholpen heeft de voorbije week. Hoewel dit op het eerste zicht geen belangrijke informatie voorstelt, valt de invloed hiervan niet te onderschatten. Klanten die op de Facebookpagina van een bedrijf kunnen aflezen dat ze gemiddeld binnen de 10 minuten geholpen worden, zullen eerder een conversatie aangaan dan wanneer ze dit helemaal niet weten.

Beide foto's samen vormen een thema dat ingesteld kan worden op een of meerdere sociale media profielen. Het instellen zelf gebeurt door het selecteren van de periode wanneer een specifiek thema actief moet zijn op een profiel. Een gebruiker kan kiezen tussen reeds ingestelde business hours (bv.maandag van 9u tot 18u) of een andere periode (bv. van november tot eind december). Zo kunnen thema's ingesteld worden om de foto's aan te passen wanneer een bedrijf open is of wanneer de kerstperiode aanbreekt. 

Na onderzoek naar verschillende bibliotheken die het mogelijk maken om tekst op afbeeldingen te plaatsen, blijkt de Fabric.js bibliotheek de beste oplossing. Met behulp van React worden verschillende componenten aangemaakt om afbeeldingen te annoteren, thema's aan te maken en toe te passen op een profiel. Ook wordt een Node.js service opgesteld die, wanneer nodig, de omslagfoto's zal genereren met aangepaste prestatie-indicatoren om deze uiteindelijk te uploaden naar Facebook of Twitter. 

Uiteindelijk werd succesvol een profile manager geïmplementeerd waarmee gebruikers op een eenvoudige manier de omslag-en profielfoto's van hun profielen kunnen beheren. 
